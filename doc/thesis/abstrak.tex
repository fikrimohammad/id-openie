%
% Halaman Abstrak
%
% @author  Andreas Febrian
% @version 1.00
%

\chapter*{Abstrak}

\vspace*{0.2cm}

\noindent \begin{tabular}{l l p{10cm}}
	Nama&: & \penulis \\
	Program Studi&: & \program \\
	Judul&: & \judul \\
\end{tabular} \\ 

\vspace*{0.5cm}

\noindent Banyaknya jumlah dokumen digital yang tersedia saat ini sudah melebihi kapasitas manusia untuk memprosesnya secara manual. Hal ini mendorong munculnya kebutuhan akan metode ekstrasi informasi (\textit{information extraction}) otomatis dari teks atau dokumen digital dari berbagai domain (\textit{open domain}). Sayangnya, sistem \textit{open domain information extraction} (\textit{open IE}) yang ada saat ini hanya berlaku untuk bahasa tertentu saja. Selain itu belum ada sistem \textit{open IE} untuk bahasa Indonesia yang dipublikasikan. Pada penelitian ini \saya memperkenalkan sebuah sistem untuk mengekstraksi relasi antar entitas dari teks bahasa Indonesia dari berbagai domain. Sistem ini menggunakan sebuah NLP \textit{pipeline}, pembangkit kandidat \textit{triple} (\textit{triple candidates generator}) dan pengembang token (\textit{token expander}) berbasis aturan serta pemilih \textit{triple} berbasis \textit{supervised learning}. Setelah melakukan \textit{cross-validation} terhadap empat kandidat model: \textit{logistic regression}, SVM, MLP dan \textit{Random Forest}, \saya menemukan bahwa \textit{Random Forest} adalah \textit{classifier} yang terbaik untuk dijadikan \textit{triple selector} denan skor F1 0.58 (\textit{precision} 0.62 dan \textit{recall} 0.58). Penyebab utama skor yang masih rendah ini adalah aturan pembangkitan kandidat yang masih sederhana dan cakupan pola \textit{dataset} yang masih rendah. \\

\vspace*{0.2cm}

\noindent Kata Kunci: \\ 
\noindent \textit{information extraction}, \textit{open domain}, \textit{natural language processing}, \textit{supervised learning}, bahasa Indonesia \\

\newpage