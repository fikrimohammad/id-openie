%
% Halaman Abstract
%
% @author  Andreas Febrian
% @version 1.00
%

\chapter*{ABSTRACT}

\vspace*{0.2cm}

\noindent \begin{tabular}{l l p{11.0cm}}
	Name&: & \penulis \\
	Program&: & \program \\
	Title&: & \judulInggris \\
\end{tabular} \\ 

\vspace*{0.5cm}

\noindent The vast amount of digital documents, that have surpassed human processing capability, calls for an automatic information extraction method from any text document regardless of their domain. Unfortunately, open domain information extraction (open IE) systems are language-specific and there is no published system for Indonesian language. This paper introduces a system to extract entity relations from Indonesian text in triple format using an NLP pipeline, rule-based candidates generator, token expander and supervised-learning-based triple selector. We cross-validate four candidates: logistic regression, SVM, MLP, Random Forest using our dataset to discover that Random Forest is the best classifier for the triple selector achieving 0.58 F1 score (0.62 precision and 0.58 recall). The low score is largely due to the simplistic candidate generation rules and the coverage of dataset.\\

\vspace*{0.2cm}

\noindent Keywords: \\ 
\noindent information extraction, open domain, natural language processing, supervised learning, Indonesian language

\newpage