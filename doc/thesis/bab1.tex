%-----------------------------------------------------------------------------%
\chapter{\babSatu}
\label{chap:babSatu}
%-----------------------------------------------------------------------------%

%-----------------------------------------------------------------------------%
\section{Latar Belakang}
%-----------------------------------------------------------------------------%
Di masa sekarang ketersediaan dokumen digital berbahasa natural seperti berita, jurnal dan buku elektronik (\textit{e-book}) sudah sangat banyak dan terus meningkat dengan cepat karena didorong oleh meningkatnya pemanfaatan komputer, \textit{smartphone} dan \textit{internet}. Jumlah dokumen digital tersebut telah melampaui batas kemampuan manusia untuk memproses secara manual sehingga menimbulkan kebutuhan akan proses otomatis untuk melakukannya \citep{banko2007open}. Salah satu proses yang dikembangkan adalah \textit{information extraction} (IE) yang secara selektif menyusun dan mengkombinasikan data yang ditemukan di dalam teks atau dokumen menjadi informasi \citep{cowie1996information}.

Meskipun \textit{IE} sudah mampu manusia untuk memproses dokumen digital dengan lebih efisien, metode yang digunakan umumnya hanya berlaku untuk kelompok dokumen yang homogen atau berada dalam satu domain (\textit{closed-domain}). Hal ini terjadi karena umumnya teknik yang dipakai dibuat sedemikian rupa untuk memanfaatkan pola tertentu pada teks atau dokumen \citep{cowie1996information}. Sebagai contoh, salah satu cara paling sederhana untuk mengekstraksi nama penulis dari berita elektronik adalah mencari nama orang di awal atau akhir dokumen. Cara yang sama tidak bisa digunakan untuk mencari nama penulis dari dokumen lain seperti jurnal karena struktur dokumen yang berbeda. Hal ini mendorong berkembangnya metode lain yang mampu mengekstraksi informasi dari berbagai domain (\textit{open domain}) yang disebut \textit{open domain information extraction} (\textit{open IE}) \citep{banko2007open}.

Seiring dengan berkembangnya waktu, beberapa sistem \textit{open IE} sudah dikembangkan untuk bahasa Inggris \citep{banko2007open,schmitz2012open,angeli2015leveraging}. Bahkan penelitian terkait melaporkan kesuksesan aplikasi \textit{open IE} untuk \textit{task} \textit{question answering} \citep{fader2011identifying} dan \textit{information retrieval} \citep{etzioni2011search}. Akan tetapi karena sistem \textit{open IE} menggunakan satu atau lebih \textit{task natural language processing} (NLP) dan aturan/heuristik yang hanya berlaku untuk bahasa tertentu, maka sistem yang dikembangkan untuk bahasa Inggris tidak dapat dipakai untuk memproses teks atau dokumen dalam bahasa lain seperti bahasa Indonesia. Oleh karena itu dalam penelitian ini \saya~memperkenalkan sistem \textit{open IE} untuk bahasa Indonesia.

\begin{figure}
	\begin{mdframed}
		\textbf{Input} \\
		"Sembungan adalah sebuah desa yang terletak di kecamatan Kejajar, kabupaten Wonosobo, Jawa Tengah, Indonesia." \\
		
		\textbf{Output} \\
		1. (Sembungan, adalah, desa) \\
		2. (Sembungan, terletak di, kecamatan Kejajar)
	\end{mdframed}
	\caption{Contoh input dan output yang diharapkan dari sistem open IE untuk bahasa Indonesia}
	\label{fig:example_io_openie}
\end{figure}

Sistem open IE yang \saya~ajukan bertujuan untuk mengekstrak sejumlah \textit{triple} (satu relasi dan dua argumen/entitas) dari satu atau lebih kalimat bahasa Indonesia seperti contoh pada Gambar \ref{fig:example_io_openie}. Sistem ini terdiri dari sebuah \textit{NLP pipeline}, pembangkit kandidat \textit{triple} (\textit{triple candidate generator}), pengembang token (\textit{token expander}) dan sebuah model \textit{supervised learning} untuk memilih \textit{triple} (\textit{triple selector}). Untuk melatih model \textit{triple selector} tersebut, \saya~juga membuat dataset berisi 1,611 kandidat \textit{triple} bahasa Indonesia yang valid dan yang tidak valid. Sistem ini diharapkan dapat menjadi referensi dalam pengembangan open IE untuk bahasa Indonesia dan juga digunakan untuk kebutuhan aplikasi yang lebih kompleks seperti pendeteksian plagiarisme, \textit{question answering} dan \textit{knowledge extraction}.

%-----------------------------------------------------------------------------%
\section{Permasalahan}
%-----------------------------------------------------------------------------%
Pada bagian ini akan dijelaskan mengenai definisi permasalahan yang ingin diselesaikan pada penelitian ini dan batasan yang ditetapkan. Definisi permasalahan diperlukan supaya arah penelitian lebih jelas dan fokus. Sedangkan batas permasalahan dibutuhkan agar penelitian ini lebih efisien serta memberikan sedikit gambaran peluang pengembangan ke depan.

%-----------------------------------------------------------------------------%
\subsection{Definisi Permasalahan}
%-----------------------------------------------------------------------------%

Permasalahan yang ditemukan dan ingin diselesaikan pada penelitian ini:

\begin{enumerate}
\item Bagaimana merancang sistem \textit{open IE} yang cocok untuk bahasa Indonesia?
\item Model \textit{supervised learning} apa yang cocok untuk \textit{open IE} bahasa Indonesia?
\item Bagaimana kinerja sistem \textit{open IE} yang dihasilkan?
\end{enumerate}

%-----------------------------------------------------------------------------%
\subsection{Batasan Permasalahan}
%-----------------------------------------------------------------------------%

Batasan permasalahan pada penelitian ini adalah:

\begin{enumerate}
	\item Pada penelitian ini \textit{Open domain information extraction} dilakukan secara otomatis dari teks bahasa Indonesia.
	
	\item Penelitian ini hanya berfokus untuk menghasilkan \textit{triple} yang eksplisit secara sintaktik. Contoh \textit{triple} yang eksplisit dari kalimat "\textit{Universitas Indonesia berada di Depok, Jawa Barat, Indonesia}" adalah \textit{(Universitas Indonesia, terletak di, Depok)}. Sedangkan \textit{triple} yang implisit seperti \textit{(Depok, terletak di, Jawa Barat)} belum ditangani pada penelitian ini.
	
	\item Proses dibatasi pada dokumen teks bahasa Indonesia yang setiap barisnya hanya berisi satu kalimat. Praproses yang dibutuhkan untuk menggubah dokumen dari format yang berbeda tidak dibahas di penelitian ini.
	
\item Penelitian ini tidak berfokus untuk mengimbangi kinerja sistem sistem \textit{open IE} untuk bahasa Inggris pada penelitian terkait.

\end{enumerate}

%-----------------------------------------------------------------------------%
\section{Tujuan dan Manfaat}
%-----------------------------------------------------------------------------%

Penyelesaian masalah yang telah dideskripsikan di atas melalui penelitian ini diharapkan dapat mencapai tujuan-tujuan dan memberikan manfaat-manfaat sebagai berikut:\\

\textbf{Tujuan}

\begin{enumerate}
	\item Merancang dan mengimplementasikan sistem \textit{open IE} untuk teks bahasa Indonesia. Hasil akhir penelitian ini bukan hanya berupa rancangan tapi juga implementasi sistem yang dapat diakses secara terbuka dan dilengkapi petunjuk penggunaan sehingga mudah digunakan.
	
	\item Mencari model \textit{supervised learning} yang sesuai sistem \textit{open IE} bahasa Indonesia. Hasil eksperimen pemilihan model dari penelitian ini diharapkan dapat memberikan gambaran mengenai karakteristik masalah dan fondasi pengembangan model ke depannya.
	
	\item Mengukur kinerja sistem \textit{open IE} secara umum berdasarkan akurasi dan waktu prosesnya. Sekalipun belum mampu menyaingi sistem \textit{open IE} bahasa Inggris yang sudah dikembangkan sejak lama, hasil pengukuran kinerja sistem ini bisa menjadi acuan untuk penelitian di masa depan.
\end{enumerate}

\textbf{Manfaat}

\begin{enumerate}
	\item Menghasilkan sistem \textit{open IE} yang dapat digunakan untuk mengekstrak entitas relasi dan argumen/entitas dalam format \textit{triple} dari teks bahasa Indonesia. Hasil ekstraksi ini dapat diaplikasikan untuk berbagai keperluan yang lebih kompleks seperti pengecekan plagiarisme, \textit{question answering}, \textit{document retrieval} .dsb.
	
	\item Memberikan acuan untuk pengembangan sistem \textit{open IE} untuk bahasa Indonesia. Sebagai salah satu penelitian pertama untuk \textit{open IE} bahasa Indonesia, diharapkan penelitian ini dapat dikembangkan lebih jauh atau menjadi referensi penelitian pada bidan terkait.
	
	\item Memberikan kontribusi terhadap perkembangan sumber daya bahasa (\textit{language resources}) Indonesia. Penelitian terkait \textit{natural language processing} (NLP) bahasa Indonesia membantu menambah sumber daya bahasa seperti \textit{dataset} dan algoritma spesifik untuk bahasa Indonesia.
\end{enumerate}

%-----------------------------------------------------------------------------%
\section{Sistematika Penulisan}
%-----------------------------------------------------------------------------%
Laporan penelitian terdiri dari lima buah bab yang mencakup pendahuluan, tinjauan pustaka, metode penelitian, hasil dan analisis serta penutup (kesimpulan dan saran). Susunan dan penjelasan sistematis dari setiap bagian dari laporan adalah sebagai berikut:

\begin{itemize}
	\item Bab 1 \babSatu \\
	Bab ini akan menjelaskan mengenai latar belakang permasalahan, rumusan masalah, tujuan, manfaat dan batasan penelitian.
	
	\item Bab 2 \babDua \\
	Bab ini akan menjelaskan landasan teori yang digunakan pada penelitian ini serta memaparkan kajian pustaka terhadap penelitian-penelitian terkait.
	
	\item Bab 3 \babTiga \\
	Bab ini akan menjelaskan mengenai tahapan, rancangan \& implementasi sistem, evaluasi dan analisis yang digunakan pada penelitian ini.
	
	\item Bab 4 \babEmpat \\
	Bab ini akan menjelaskan tentang hasil eksperimen dan analisis hasil eksperimen.
	
	\item Bab 5 \babLima \\
	Bab ini akan menjelaskan tentang kesimpulan dari penelitian yang telah dilakukan dan saran untuk penelitian berikutnya.
\end{itemize}

