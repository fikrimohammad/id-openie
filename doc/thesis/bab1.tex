%-----------------------------------------------------------------------------%
\chapter{\babSatu}
%-----------------------------------------------------------------------------%
\todo{tambahkan kata-kata pengantar bab 1 disini}


%-----------------------------------------------------------------------------%
\section{Latar Belakang}
%-----------------------------------------------------------------------------%
Ketersediaan dokumen bahasa Indonesia dalam jumlah banyak dengan domain yang beragam menuntut proses ekstraksi yang otomatis \& generik \\

Kebutuhan akan representasi dokumen teks bahasa Indonesia yang mencerminkan informasi lintas domain di dalamnya. Contoh aplikasinya adalah deteksi plagiarisme, document retrieval .dsb \\

Pembangunan knowledge base bahasa Indonesia dengan ontologi yang matang dapat dimulai dari ekstraksi informasi lintas domain \\

\begin{table}
	\centering
	\caption{Perbandingan antara \f{information extraction} tradisional, \f{open domain extraction} dan \f{knowledge extraction}}
	\label{table_paradigm_comparison}
	\begin{tabular}{|l|c|c|c|}
		\hline 
		 & \textbf{IE} & \textbf{Open IE} & \textbf{KE} \\ 
		\hline 
		\textbf{Domain} & Closed & Open & Open \\ 
		\textbf{Format} & Depends on domain & Triples & RDF Triples \\ 
		\textbf{Ontology} & Not available & Optional & Mandatory \\ 
		\hline 
	\end{tabular} 
\end{table}

%-----------------------------------------------------------------------------%
\section{Permasalahan}
%-----------------------------------------------------------------------------%
Pada bagian ini akan dijelaskan mengenai definisi permasalahan 
yang \saya~hadapi dan ingin diselesaikan serta asumsi dan batasan 
yang digunakan dalam menyelesaikannya.


%-----------------------------------------------------------------------------%
\subsection{Definisi Permasalahan}
%-----------------------------------------------------------------------------%
\todo{Tuliskan permasalahan yang ingin diselesaikan. Bisa juga
	berbentuk pertanyaan}


%-----------------------------------------------------------------------------%
\subsection{Batasan Permasalahan}
%-----------------------------------------------------------------------------%

\begin{enumerate}
\item Ekstraksi dilakukan pada teks bahasa Indonesia yang telah melewati praproses untuk memisahkan tiap kalimat menjadi satu baris
\item Hanya mengkstraksi triples yang eksplisit secara struktur dependency relation. Contoh: Universitas Indonesia berada di Depok, Jawa Barat, Indonesia → (Universitas Indonesia-terletak di-Depok)
\item Dataset yang dikembangkan dalam penelitian ini hanya dataset untuk triple selector sedangkan dataset lainnya menggunakan yang sudah tersedia
\end{enumerate}

%-----------------------------------------------------------------------------%
\section{Tujuan dan Manfaat}
%-----------------------------------------------------------------------------%

Penelitian ini bertujuan untuk:
\begin{enumerate}
\item Menghasilkan kakas open domain information extraction otomatis untuk teks bahasa Indonesia
\item Mendefinisikan model open domain information extraction untuk teks bahasa Indonesia
\item Menentukan aturan-aturan (rules) yang dibutuhkan model
\item Menentukan fitur-fitur yang dapat digunakan pada model
\item Menentukan hyperparameter yang optimal untuk model
\end{enumerate}

Manfaat yang diharapkan dari penelitian ini adalah:
\begin{enumerate}
\item Menyediakan kakas yang menghasilkan representasi dan/atau informasi yang dapat dimanfaatkan untuk aplikasi lain
\item Memberikan informasi mengenai model open domain information extraction untuk bahasa Indonesia
\item Mendorong pengembangan sumber daya (language resources) bahasa Indonesia
\end{enumerate}

%-----------------------------------------------------------------------------%
\section{Metodologi Penelitian}
%-----------------------------------------------------------------------------%
\todo{Tuliskan metodologi penelitian yang digunakan.}


%-----------------------------------------------------------------------------%
\section{Sistematika Penulisan}
%-----------------------------------------------------------------------------%
Sistematika penulisan laporan adalah sebagai berikut:
\begin{itemize}
	\item Bab 1 \babSatu \\
	\item Bab 2 \babDua \\
	\item Bab 3 \babTiga \\
	\item Bab 4 \babEmpat \\
	\item Bab 5 \babLima \\
\end{itemize}

\todo{Tambahkan penjelasan singkat mengenai isi masing-masing bab.}

