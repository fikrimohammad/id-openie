%-----------------------------------------------------------------------------%
\chapter{\babLima}
%-----------------------------------------------------------------------------%

Pada bab ini dijelaskan kesimpulan penelitian ini dan saran untuk pengembangan penelitian di masa depan.

%-----------------------------------------------------------------------------%
\section{Kesimpulan}
%-----------------------------------------------------------------------------%

This paper introduces an open domain information extraction system for Indonesian text using basic NLP pipelines and combination of heuristics and machine learning models. The system is able to extract meaningful domain-independent relations from Indonesian sentences to be used as document representation or document understanding task. Additionally, the source code and datasets are published openly\footnote{Paper source code \url{https://github.com/yohanesgultom/id-openie}} to improve research reproducibility.

%-----------------------------------------------------------------------------%
\section{Saran}
%-----------------------------------------------------------------------------%

In the future, we plan to improve the performance of our system finding better heuristics for triple candidates generator to reduce the negative samples. We also plan adding more training data for triple selector to improve the precision and recall score. We also need to create dataset for triple candidates generator and token expander in order to properly evaluate further improvement of both components. We also consider adding confidence level in the output of every phases (NLP pipelines, candidate generator, triple selector, token expander) and including them as features and/or heuristics may also improve the overall performance of the system.